\documentclass[a4paper,11pt]{article}
\usepackage[utf8]{inputenc}
\usepackage[T1]{fontenc}
\usepackage[french]{babel}
\usepackage{makeidx}
\usepackage{textcomp}
\usepackage{graphicx}
\usepackage{mathtools,amssymb,amsthm}
\usepackage{lmodern}
\usepackage{multirow}
\usepackage{listings}
\usepackage{array}
\usepackage{longtable}

\title{Projet simulation - Rapport}
\author{Maxime Gonthier (21500231) - Benjamin Guillot (21500545)}

\begin{document}
\pagenumbering{gobble}\clearpage
\maketitle

\newpage
\tableofcontents

\newpage
\section{Introduction}
	L'objectif de ce projet est de simuler en temps discret des arrivées et des services dans un cyber café. On en déduira des mesures d'évaluations à l'aide du calcul du temps moyen d'attente et du 90ème percentile du temps d'attente. 
	
\section{Explication de la programmation}
	\subsection{Entrée et sortie du progamme}
	En entrée le programme prend un fichier texte contenant les données que l'on veut faire varier dans notre application (ici lambda).
	On obtient en sortie 5 fichier :
	\begin{itemize}
		\item Result\_modele1.txt
		\item Result\_modele2.txt
		\item Result\_modele3.txt
		\item resultE.txt
		\item result90.txt
	\end{itemize}
	Les 3 premiers fichiers contiennent pour chaque valeurs de lambda le temps moyen d'attente $E[A]$ et le 90 percentile du temps d'attente $t_{90}E[A]$.\\
	Les deux derniers fichiers sont utilisés pour l'affichages des courbes obtenues après la simulation.
	\subsection{structure du progamme}
	Nous avons choisit pour la programmation de gérés les différents modèles dans les fonctions $Arrivee\_Client$ et $service\_event$. Ces deux fonctions on 3 modes de fonctionnement
	passé en argument pour savoir quel modèle on est en train de simuler. Il y a un simulateur par modèles.
	Il sont appelés dans le main avec en argument :
	\begin{itemize}
		\item un fichier dans lequel écrire les données relatives a la simulation.
		\item la valeur de lambda en train d'être testée.
	\end{itemize}
	\subsubsection{premier modèle}
		le premier modèle représente une M/M/N classique. on traite donc les clients dès qu'un serveur est libre.
	\subsubsection{second modèle}
		Le second modèle peut être simulé avec une M/M/1. On ajoute une condition aux arrivées de client qui est un pourcentage de chance d'arriver dans la file, comme on a 10 serveurs en theorie, un client arrive dans une file avec une probabilité $\frac{1}{10}$.
	


\section{Explication des résultats théoriques}
	
\section{Conclusion}
	
\end{document}
